\documentclass[12pt]{article}%12-point font
\usepackage[utf8]{inputenc}
\usepackage{amsmath}
\usepackage{mathabx}
\usepackage{graphicx}
\usepackage{siunitx}
\usepackage{commath}
\usepackage{xcolor}
\usepackage{hyperref}
\usepackage{tkz-euclide} %  https://www.mathcha.io/ generates TeX figures 
\usepackage{braket} 
\usepackage{endnotes}
\usepackage{subcaption}
\usepackage{appendix}
\usepackage{float}

\linespread{1.5}%this sets the spacing to 1.5
\usepackage{fullpage}%this sets the margins to 1 inch

\begin{document}


\begin{center}
\begin{Large}
\textbf{Memo for Ultrasonics Term Project}\\
\end{Large}
\textbf{Chirag Gokani, Yuqi Meng}

\end{center}


\section*{Topic} 

The pressure field of a vortex beam generated by a single-element transducer and phase plate will be found, replicating the theoretical results presented in \cite{ref1} by Terzi et al. Their discussion will be extended with Fresnel and Fraunhofer analyses, for which analytical expressions of the pressure magnitude and phase are sought. Applications of vortex beams will also be described. If time permits, a finite element method (FEM) solution will be sought to validate the analysis. A parameter optimization may also be discussed.



\section*{Preliminary considerations}
Vortex beams are sound waves carrying angular momentum. Their radiation patterns projected on a transverse have vanishing on-axis intensity. In this discussion, they are described in the time domain with the $e^{-i \omega t}$ time convention. The transducer and phase plate generating the vortex beam lie in the $x$-$y$ plane, and the beam propagates in the $+z$-direction. The polar angle is given by $\varphi = \arctan (y/x)$. The wave equation shows that the vorticity contributes a factor of ${e}^{i\Phi(\varphi) } = {{e}^{il\varphi }}$ to the complex-exponential form of solution, where $l = 0, \pm1, \pm2,\dots$is the orbital number. The $l=0$ mode has no vorticity, the $l= \pm1$ mode features one equal-phase surface, and higher orbital numbers each correspond to a family of $l$ helicoids. The time-independent pressure for $z> 0$ therefore has the form $p(\varphi,z)e^{i\Phi(\varphi)}$. The magnitude at an arbitrary location is found using the Rayleigh integral, which amounts to taking a surface integral given the velocity source condition. Meanwhile, the phase is found using a forward propagation algorithm outlined in \cite{ref2} and \cite{ref5}.



\section*{Work distribution}

C.G. and Y.M. will work together to calculate the pressure amplitude and phase distributions given by equations (1) and (2) in Terzi et al. The replication of Figure 2(a) and 2(b) will be a group effort. If time permits, C.G. and Y.M. will work together to seek an FEM solution to the problem.

C.G. will evaluate the Fraunhofer and Fresnel limits and seek analytical solutions for the pressure amplitude and phase distributions. C.G. will type the supplementary notes.

Y.M. will research the applications of vortex beams and their relation to the larger field of ultrasonics. If time permits, Y.M. will modify the geometry and other relevant parameters in the FEM model to optimize its design for applications.

C.G. and Y.M. will both contribute the respective sections to the report. The creation of the presentation is straightforward as it is largely a cut-and-paste effort from the report, and Chirag will lead this effort in \TeX.


\section*{Logistics \& Timeline} 

We will use \textbf{\href{https://github.com/cag170030/Ultrasonics_2022}{GitHub}} for code and \href{https://www.overleaf.com/read/rryhckcywhks}{\textbf{Overleaf}} for the report \& presentation. 

\begin{itemize}
    \item 4/1: C.G. and Y.M. write MATLAB script for Terzi et al. equation (1). C.G. evaluates Fraunhofer limit of the analytical problem. C.G. completes supplemental notes.
    \item 4/8: C.G. and Y.M. replicate Terzi figure 2(a). C.G. evaluates Fresnel limit of the analytical problem.
    \item 4/15: C.G. and Y.M. formulate code for Terzi et al. equation (2). Y.M. researches applications of vortex beams.
    \item 4/22: C.G. and Y.M. replicate Terzi figure 2(b).
    \item 4/29: C.G. and Y.M. pursue FEM model if time permits.  
    \item 5/2: C.G. and Y.M. complete term paper and submit on Canvas; C.G. moves relevant parts of paper into \texttt{beamer} slides for presentation. 
    \item 5/3 or 5/5: C.G. and Y.M. meet to practice presentation and present to class.
\end{itemize}



\section*{Potential obstacles}

\begin{itemize}
    \item Deriving Terzi et al. equation (2)
    \item Performing integration for equation (2)
    \item Arriving at analytical solutions in the Fresnel and Fraunhofer limits (may not be possible; integral may diverge if not enough terms are retained)
    \item FEM model might not happen\dots it will depend on how smoothly we can complete the primary aspects of the project
\end{itemize}




\begin{thebibliography}{99}
\bibitem{ref1}
Terzi, M. E., Tsysar, S. A., Yuldashev, P. V., Karzova, M. M., and Sapozhnikov, O. A. \href{https://doi.org/10.3103/S0027134916050180}{Generation of a vortex ultrasonic beam with a phase plate with an angular dependence of the thickness}. \textit{Moscow University Physics Bulletin}. \textbf{72}, 61–67 (2017).


\bibitem{ref2}
Melde, K., Mark, A. G., Qiu, T., and Fischer, P. \href{https://doi.org/10.1038/nature19755}{Holograms for acoustics}. \textit{Nature}. \textbf{537}, 518–522 (2016).

\bibitem{ref3} 
Sallam, A., Meesala, V.C., Hajj, M.R., et al. \href{https://doi.org/10.1063/5.0065489}{Holographic mirrors for spatial ultrasound modulation in contactless acoustic energy transfer systems}.  \textit{Applied Physics Letters}. \textbf{119}, 144101 (2021).

\bibitem{ref4} 
Parker, S. \href{https://hdl.handle.net/2152/87049}{Physical limitations and practical considerations for the creation of acoustic holograms}. M.S. Thesis, The University of Texas at Austin. (2020).

\bibitem{ref5} 
Sapozhnikov, O.A., Tsysar, S.A., et al. \href{https://doi.org/10.1121/1.4928396}{Acoustic holography as a metrological tool for characterizing medical ultrasound sources and fields}. \textit{The Journal of the Acoustical Society of America} \textbf{138}, 1515 (2015). 


\end{thebibliography}
\end{document}


